\chapter*{Conclusion}
\pagestyle{fancy}\lhead{\textbf \footnotesize\it{Conclusion}}
\pagestyle{fancy}\chead{} \pagestyle{fancy}\rhead{}
\pagestyle{fancy}\cfoot{} \pagestyle{fancy}\rfoot{\thepage}
%\addcontentsline{toc}{chapter}{General conclusion}
%%%%%%%%%%%%%%%%%%%%%%%%%%%%%%%%%%%% 
This thesis has investigated the intersection of natural language processing technologies and the legal domain, with a particular focus on the capabilities and limitations of large-scale language models in processing complex Arabic legal texts. We began by presenting a foundational overview of Large Language Models (LLMs), exploring their potential for legal text understanding and generation, while also highlighting the specific challenges they face in dynamic, high-stakes environments. Following this, we introduced the concept of Agentic AI, outlining its general characteristics and its emerging role in enabling autonomous, goal-driven reasoning.

To address the limitations of static language models, we turned to RAG architectures, which enhance language models by integrating external document retrieval systems. This hybrid approach offers a more grounded and context-aware solution, crucial for legal question answering and reasoning tasks

Chapter 3 addressed one of the core limitations of traditional RAG systems: the use of fixed top-k document selection. We proposed a dynamic candidate selection mechanism that adjusts the retrieval scope based on the complexity of the user query, thereby improving retrieval precision and reducing noise in the generation process.

In Chapter 4, these concepts were brought together in a fully implemented Arabic RAG pipeline. This included preprocessing, semantic embedding, FAISS-based indexing, and generation using a fine-tuned model. The resulting conversational agent enables natural language querying of Algerian legal content and delivers grounded, contextually relevant answers. Additionally, the thesis introduced a curated dataset of Algerian legal cases in Arabic, addressing a major gap in Arabic legal NLP resources.

Overall, this work offers both theoretical contributions and practical tools for advancing legal AI in under-resourced settings. By combining dynamic retrieval, generation, and agentic reasoning, the thesis lays the groundwork for the next generation of intelligent legal assistants in the Arab world.

%\newpage






